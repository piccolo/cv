\begin{rubric}{Expériences Professionelles}
\subrubric{Ingénieur Recherche et Développement - ULTEO SAS - Colombelles}
\entry*[2011-]
\textit{Optimisation des performances réseaux par une approche traitement d'images}
\par Ce travail consiste en la réalistation de codecs de compression et de décompression de données
afin de réduire la taille des images transitant dans le réseau en tenant compte de différents aspects :
\begin{itemize}
\item la bande passante disponible afin d'éviter la saturation du réseau;
\item l'exécution temps réelles de ces algorithmes afin d'éviter la latence;  
\item la scalabilté de ces algorithmes permettant ainsi de lancer plusieurs sessions concurrentes.
\end{itemize}

\subrubric{Stages Post-doctoraux} 
\entry*[2010-2011]
\textit{Redressement d'images d'empreintes digitales par projection de lumière structurée.}
\par Ce travail, en collaboration avec la police judiciaire et la gendarmerie consiste 
en l'extraction d'informations présentent dans une photo et en son redressement à l'aide de lumière structurée et d'outils mathématiques avancées : 
\begin{itemize}
\item la mise en équation de notre modèle afin de se rapprocher au plus près des données expérimentales;
\item la minimisation de modèles pour déterminer les paramètres du modèle;
\item la reconstruction 3D afin de permettre le redressement de l'empreinte.
\end{itemize}
\entry*[2009-2010]
\textit{Segmentation d'images de lames histologiques de cancer du sein.}
\par Ce travail, en collaboration avec des chercheurs hospitaliers et différentes équipes de recherches,
consiste en l'élaboration d'un outils permettant d'extraire et d'identifier les différentes structures 
présentes dans des images histologiques multi-résolutions, à l'aide d'outils mathématique avancées :
\begin{itemize}
\item la régularisation d'image pour uniformiser de façon adaptative l'image;
\item la théorie des graphes permettant de modéliser notre image;
\item la quantification spectrale pour extraire et identifier nos données;
\end{itemize} 

\subrubric{Thèse de Doctorat}
\entry*[2004--2008]
\textit{Classification floue et modélisation IRM}
\par Ce travail a pour finalité le développement de logiciels d'aide au diagnostic
des maladies liées à l'obésité utilisant des outils algorithmique et mathématique : 
\begin{itemize}
\item la segmentation d'image afin d'extraire les structures;
\item la classification afin d'identifier les structures extraites à partir de critères morphologiques, colorimétriques et empiriques (fournies par des experts);
\item l'apprentissage automatique (SVM) afin de rendre automatique la classification.
\end{itemize}    
%\subrubric{Stages} 
% \entry*[03-09 2003] 
%\textit{Restauration d'images fondée sur les EDP fractionnaires.}
%\entry*[07-09 2002]
%\textit{Moteur de rendu non photoréaliste.}

%\subrubric{Ingénieur Développement - NetworkSystem - Paris} 
%\entry*[06-11 2001]
%\textit{Développement d'un moteur 3D pour le web}


\end{rubric}
