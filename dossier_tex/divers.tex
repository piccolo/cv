\begin{rubric}{Divers}

  \entry*[Communications] 6 journaux et 22 conférences
  (internationnales et nationales).

  \entry*[Logiciels] 3 logiciels multiplateformes~: 
  \begin{itemize}
  \item Mesure de la graisse viscérale et sous-cutanée en IRM (C++).
  \item Mesure de la stéatose hépatique en IRM (C++).
  \item Proportion de stéatose sur des biopsies hépatiques en imagerie
    histologique (Java)
  \end{itemize}

  \entry*[Transmission]  Plus de 500 h d'enseignements en informatique et
  mathématiques devant un public hétérogène. 

  \entry*[Vulgarisation] Présentations du traitement d'images à la \emph{Fête de la Science}. 
  
  \entry*[Langues étrangères] \textbf{Anglais} : Courant - pratique en environnement professionnel - \textbf{Allemand} : Notions - \textbf{Italien} : Notions
  % \entry*[2009-2011] Participation à la \emph{Fête de la Science} par des
  % interventions dans les lycées de la région afin de faire connaître le
  % métier de chercheur et le traitement d'images.

  % \entry*[2003-2011] Collaborations avec plusieurs laboratoires de
  % recherche : GREYC (Université de Caen-ENSICAEN), L3I (Université de La Rochelle), LISA (Université
  % d'Angers), ETIS (Université de Cergy-Pontoise)

  % \entry*[Janv 2010] 
  % Comité d'organisation du 17\up{ième} congrès francophone AFRIF-AFIA à Caen

  % \entry*[2005-2007] 
  % Représentant des Doctorants au Conseil de Laboratoire LISA, Université d'Angers

  % \entry*[Juin 2004]
  % Coordinateur Logistique à la Semaine du Document Numérique, organisée par le laboratoire L3I (Informatique, Image et Interaction), Université de La Rochelle.

  % \entry*[2003-2004] 
  % Représentant des étudiants de DEA de l’Université de La Rochelle au Conseil de l’Ecole Doctorale (2003-2004).

  % \entry*[2000-2001] 
  % Représentant des étudiants de Licence d’informatique au Conseil d’UFR de l’Université de Caen (2000-2001)

\end{rubric}
