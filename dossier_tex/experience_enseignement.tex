

\begin{rubric}{Expérience d'Enseignement}
\subrubric{Enseignements (CM, TD, TP)}
\entry*[2010-2011]
\textbf{Vacataire (50h ETD) à l'Ensicaen}
\begin{itemize} 
\item Outil Développement Logiciels (make, doxygen, bash);
\item Multimédia (Gimp, Inkscape, Blender, VRML, Flash)
\item Compilation (flex,bison);
\end{itemize}
\entry*[]\textbf{Vacataire (50h ETD) à l'IUT de Saint-Lô}
\begin{itemize} 
\item PHP/Base de données;
\item Système d'information et base de données;
\end{itemize}


\entry*[2009-2010]
\textbf{Vacataire (102h ETD) à l'Ensicaen}
\begin{itemize} 
\item Algorithmique Avancée;
\item PHP/Base de données;
\item Technologie Internet I (HTML);
\item Technologie de l'information (XML, XSL, XSLT, \ldots)
%\item Encadrement de projet : 1\up{ière} et 2\up{ième} année de cycle ingénieur.
\end{itemize}

\entry*[2008-2009]
\textbf{ATER à temps complet (192h ETD) (61\up{ième} section)} à l'\'Ecole d'Ingénieur d'Angers (ISTIA) en :
\begin{itemize}
\item Informatique : Administration de réseau sous Windows;
\item Algorithmique;
\item Mathématique : Transformée de Laplace, de Fourier, en Z;
%\item Encadrement de projet : 2\up{ième} année de cycle ingénieur.
\end{itemize}

\entry*[2005-2007]
\textbf{Vacataire (128h ETD) à IMIS-ESTHUA de l'Université d'Angers}
\begin{itemize}
\item Programmation VB;
\item Initiation à la bureautique;
\item Initiation à l'algorithmique.
\end{itemize}

\entry*[2003-2004]
\textbf{Vacataire (27h ETD) à l'Université de La Rochelle}
\begin{itemize}
\item Programmation Orienté Objet (C++) et Interface Homme Machine;
\item Intitiation à la programmation C.
\end{itemize}

\subrubric{Encadrements de projets et stages}
\entry*[2010-2011]
\textbf{Projet 2\up{ième} année ENSICAEN} : Création d'un moteur de jeux sur plateforme mobile Androïd;
\textbf{Projet 3\up{ième} année ENSICAEN} : Transcription d'un bibliothéque DirectX en OpenGL en collaboration avec l'entreprise Wyzarbox;


\entry*[2009-2010]
\textbf{Projet 2\up{ième} année ENSICAEN} : Développement d'une application client-serveur pour l'interogation de serveur d'images médicale.
\entry*[2009-2010]
\textbf{Projet 1\up{ière} année ENSICAEN} : Réaliser une application
de réalité augmentée à l'aide de la bibliothèque ARToolKit.

\entry*[03-08/2009]
\textbf{2 stages de Master Recherche Signaux et Images en Biologie et Medecine de l'Université d'Angers} : 
\begin{itemize}
  \item Utilisation de la programmation GPU pour le traitement des grandes images.
  \item Recalage et mesure de vélocimétrie en IRM
\end{itemize}

\entry*[03-08/2009]
\textbf{Stage de Master Recherche Système Dynamique et Signaux  de l'ISTIA (\'Ecole d'ingénieur de l'Université d'Angers} : Utilisation du GPU pour la calcul par intervalle.

\entry*[2005-2008]
\textbf{3 stages de 3\up{ième} année d'école d'ingénieur ESEO} : Méthodes de segmentation en traitement d'image médicales.

\subrubric{Divers:}

\entry*[2009-2010]
Présentations des activités de l'équipe Image du laboratoie GREYC et
initiation au traitement d'images dans plusieurs lycées du Calvados
(Lycée Malherbe et lycée Victor Hugo de Caen et lycée Alain Chartier de Bayeux).

\end{rubric}
