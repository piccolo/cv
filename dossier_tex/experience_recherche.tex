\begin{rubric}{Expérience de Recherche}
  \subrubric{Stage Post-doctoral} \entry*[2010-2011]
  \textit{Reconstruction d'objets 3D par lumière structurée}\newline

  \textit{Résumé} : Mon second stage post-doctoral au sein de l'équipe
  Image du GREYC se déroule dans le cadre du projet ANR CARTES
  (Capture Aisée et Rapide de Traces et d'Empreintes sur Scènes). Les
  techniciens de la gendarmerie et de la police judiciaire doivent
  quotidiennement prélever des traces sur des scènes de crimes. Lors de
  ce prélèvement, la trace peut être endommager ou dégrader par le
  technicien à cause de la nature du support, d'une fausse
  manipulation. L'objectif de ce projet ANR est de proposer un outil
  permettant d'acquérir facilement et rapidement la trace et dans le
  cas d'une trace sur une surface courbe, pouvoir redresser cette
  empreinte et ainsi rendre son identification plus aisée. L'objectif
  de mon travail est d'extraire la trace et de corriger les
  déformations géométriques dûes à la forme du support. Pour cela,
  nous projetons une lumière structurée sur l'empreinte et nous
  étudions le déphasage des frange pour reconstruire une carte de
  profondeur. Il faut également différencier l'empreinte de la lumière
  projetée. Pour cela, nous utilisons une approche variationnelle permettant
  d'extraire l'empreinte puisque que nous avons une connaissance
  précise de la lumière projetée.

  \subrubric{Stage Post-doctoral}
  \entry*[2009-2010]
  \textit{Segmentation d'images histologiques par approche multi-resolution}\newline

  \textit{Résumé} : Mon stage post-doctoral au sein du GREYC, dans
  l'équipe Image dans le cadre du projet ANR FOGRIMMI porte sur la
  fouille dans de grandes images microscopiques. Les pathologistes
  utilisent leur outil de travail, le microscope, depuis le 17ème
  siècle et la plus grande partie de leur analyse est visuelle. C'est
  pourquoi l'imagerie pathologique est devenue un environnement
  d'imagerie médicale d'importance grandissante qui présente de
  nombreux défis. La toute dernière génération de stations d'imagerie
  pathologique permet de disposer d'interfaces pour la microscopie
  virtuelle par l'Imagerie d'une Lame Entière (ILE). L'ILE produit
  cependant des images énormes pour une unique lame non compressée
  (environ 30 Go) et la gestion de telles masses de données représente
  un réel défi de cette nouvelle ère de la microscopie numérique. Les
  images d'ILE sont représentées par un seul fichier contenant l'image
  à pleine résolution et par une séquence d'images codant différentes
  résolutions de l'image de base. Ce type d'images a donc la
  particularité de faire apparaître différentes structures à
  différentes résolutions. Ces images mêlent de façon intrinsèque une
  représentation multi-résolution et multi-échelle. Ce projet de
  recherche est consacré à la conception et à la validation d'un
  modèle permettant de coder les différentes structures contenues dans
  de telles images à leur résolution intrinsèque. Le modèle sera
  validé à l'aide de deux applications (en histologie et en cytologie)
  sur des images d'ILE en collaboration avec le centre anti-cancéreux
  François Baclesse.  Mon travail durant ce stage est de segmenter des
  images histologiques de biopsie mammaire et d'extraire et
  d'identifier des cellules de mitose. Pour cela, nous avons mis en
  place une méthode de segmentation fondée sur l'utilisation d'un
  focus d'attention. Ce focus d'attention est obtenu par les
  segmentations successives utilisant les graphes effectuées niveau
  par niveau.
    
  \subrubric{Thèse de Doctorat}
  \entry*[2004--2008]
   \textbf{Doctorat en Traitements d'Images}
  \textit{Classification floue et modélisation IRM : Application à la quantification de la graisse pour une évaluation optimale des risques pathologiques associés à l'obésité}\newline
  
  \textit{Résumé}: Les travaux présentés dans cette thèse traitent de
  l’apport de l’incertitude, de l’imprécision et de l’a priori en
  traitement d’images médicales, dans le cadre d’outils d’aide au
  diagnostic des pathologies conséquentes de l’obésité et du
  surpoids. Deux parties composent ce travail : une modélisation du
  signal IRM d’une séquence prototype fournie par GE, et une méthode
  de classification floue adaptée pour répondre aux attentes des
  experts radiologistes et anatomopathologistes. Le signal IRM est
  issu des différents constituants du voxel. Afin de déterminer la
  proportion de graisse dans le tissu, les signaux issus de l’eau et
  de la graisse sont déterminées par régression à partir des images
  IRM obtenues en prenant en compte un a priori sur le bruit présent
  sur les images. Considéré de Gauss sur les images réelles et
  imaginaires, et de Rice sur les images amplitudes, cet \emph{a-priori} sur
  le bruit a permis de mettre en évidence l’apport de l’utilisation
  des données brutes lors de la quantification de la proportion de
  graisse et d’eau par rapport à une quantification uniquement
  effectuée sur les données amplitudes. La méthode de classification
  présentée ici permet une dépendance à longue distance lors du calcul
  des centroïdes. Cette méthode combinée à un algorithme de
  connectivité floue est adaptée à la mesure de la graisse viscérale
  et sous-cutanée. Elle fut également utilisée pour la quantification
  des vacuoles de triglycérides présentes sur des biopsies
  hépatiques. De part la proportion très hétérogène des vacuoles de
  stéatose, fonction du degré de la pathologie, nous avons amélioré
  l’algorithme de classification par une supervision permettant
  d’orienter la classification afin de se dédouaner de cette
  hétérogénéité. La classification est ensuite combinée à un système
  expert permettant d’éliminer les erreurs de classification
  survenues. L’ensemble des méthodes fut évalué dans le cadre
  d’expérimentations animales et de différents protocoles de recherche
  clinique. Ce travail de thèse est valorisé par la création de
  logiciels actuellement utilisés en routine clinique.\newline

%\textit{Mots clés : classification floue, connectivité floue, fusion floue, segmentation d’images, système expert, IRM, image histologique, aide au diagnostic}
  
  \subrubric{Stage de recherche - DEA}
  \entry*[04-09/2004]
  \textbf{Doctorat en Traitements d'Images}
  \textit{Contribution à un problème ouvert : la dépendance longue portée en restauration d’images} \newline

   \textit{Résumé}: L’objectif de ce stage était de répondre à une
   problématique récente, la dépendance longue portée en restauration
   d’images, en utilisant un outil mathématique récent, les EDPs
   fractionnaires. Cet outil mathématique n’a encore jamais été
   utilisé en traitement d’images. Ce stage a permis de mettre en
   place une chaîne de traitements répondant à cette problématique et
   reposant sur une coopération EDP/classification. Un prototype de
   logiciel fut programmé en C++ et Gtk (pour l’interface). Cette
   méthode améliore la restauration des images par rapport aux
   méthodes classiques (chaleur, Alvarez, etc.) en conservant les
   contours, car on n’utilise plus un voisinage local de l’image mais
   un voisinage éloigné (connexe ou non).


\end{rubric}
